\documentclass[10pt,reqno]{amsart}
\usepackage{amsmath}
\usepackage{amsthm}
\usepackage{amssymb}
\usepackage{amsfonts}
\usepackage{latexsym}
\usepackage{verbatim}
\usepackage{graphicx}

\newtheorem*{Remarks}{Remarks}
\newtheorem*{Theorem}{Theorem}
\newtheorem*{Definition}{Definition}



\begin{document}
\begin{center}
\huge Topology Notes, 2 April 2010 \\
\normalsize
Elaine Brow
\end{center}
\vspace{.1in}
\begin{center}
{\it "Homotopy is like the most important concept for the rest of the semester."}
\end{center}
\vspace{.1in}
First, a couple of  pictures.\\
$f_0: S^1 \rightarrow \mathbb{R}^2$ (such that the image looks like a circle)\\
$f_1: S^1 \rightarrow \mathbb{R}^2$ (such that the image looks like a square)\\
\vspace{1in}

\noindent It is visually apparent that one image can be warped to make the other, so $f_0$ and $f_1$ are homotopic.\\\\
Homotopic functions need not be homeomorphic.\\
These  are homotopic, for example (a circle space and an infinity-shaped space).\\
\vspace{1in}

\noindent{\bf Example.}\\
Let $X = I$ and $Y = \mathbb{R}^2$\\
Define\\
$f_0: I \rightarrow \mathbb{R}^2$ by $f_0(x) = (x,0)$,\\
$f_1: I \rightarrow \mathbb{R}^2$ by $f_1(x) = (x,x^2)$, and\\
$F: I \times I \rightarrow \mathbb{R}^2$ by $F(x, t) = t(x, x^2) + (1-t)(x, 0)$.\\
Observe that $F$ is continuous.\\
Note also that $F(x,0) = (x, 0) = f_0(x)$ and $F(x,1) = (x, x^2) = f_1(x)$.\\
Thus $F$ is a homotopy.\\\\
How to draw a homotopy:\\
\vspace{1in}

\noindent The image of a vertical segment in the square is the path taken  by the $x$ coordinate of that segment during the homotopy.\\\\
\noindent{\bf Remarks.}
\begin{enumerate}
\item Suppose $Y$ is not path connected and $f_0(x)$  and $f_1(x)$ are in different path components. Then there does not exist a homotopy from $f_0$ to $f_1$.\\
\vspace{1in}
\item If $Y$ is path connected and $f_0$ and $f_1$ are paths in $Y$, then $f_0 \simeq f_1$. (Homotop (the verb!) $f_0$ to its initial point, move it to the initial point of $f_1$ and then stretch it back out into $f_1$.)
\vspace{1in}
\end{enumerate}
{\bf Example.}\\
Let $Y = \mathbb{R}^2 - \{(\frac{1}{2}, \frac{1}{3})\}$\\
Define $f_0: I \rightarrow Y$ by $f_0(s) = (s,s^2)$, and $f_1: I \rightarrow Y$ by $f_1(s) = (s,s).$\\
Because there's a hole in $Y$, we won't be able to just bend $f_0$ over to $f_1$..\\\\
Define $G: I \times I \rightarrow Y$  by $G(s,t) = f_0((1-t)s)$.\\
Observe that $G$ is continuous, and $G(s,0) = f_0(s)$ and $G(s,1) = f_0(0)$\\\\
Now define $H: I \times I \rightarrow Y$ by $H(s,t) = f_1(ts)$.\\
Observe that $H$ is continuous and $H(s,0) = f_1(0)$ and $H(s,1) = f_1(s)$.\\\\
Finally, define $F: I \times I \rightarrow Y$ by
\begin{displaymath}
   F(s, t) = \left\{
     \begin{array}{lr}
       G(s, 2t) & : t \in [0, \frac{1}{2}]\\
       H(s, 2t-1) & : t \in [\frac{1}{2}, 1]
     \end{array}
   \right. .
\end{displaymath}
\vspace{.9in}

\noindent\underline {$F$ is continuous.}\\
Let $A = I \times [0, \frac{1}{2}]$ and $B = I \times [\frac{1}{2}, 1]$. Both are closed in $I\times I$.\\
$A \cap B = I \times \{\frac{1}{2}\}$.\\
$G(s, 2(\frac{1}{2})) = G(s, 1) = f_0(0)  = (0,0)$, and\\
$H(s, 2(\frac{1}{2})-1) = H(s, 0) = f_1(0)  = (0,0)$.\\
Since $G$ and $H$ are continuous and agree at $A\cap B$, by the Pasting Lemma, $F$ is continuous.\\\\
\underline {$F$ is a homotopy.}\\
$F(s, 0) = G(s, 0) = f_0(s)$.\\
$F(s, 1) = H(s, 1) = f_1(s)$.\\
Thus $F$ is indeed an homotopy.\\\\
So $f_0$ is homotopic to $f_1$!\\\\
{\bf Question.}\\
What path does $(1, 1)$ take during the aforementioned homotopy?\\
\vspace{1in}

\noindent {\bf Answer.}\\
Informally put, it moves down along $f_0$ and climbs back up $f_1$ to its old position.\\
More formally,\\
\begin{displaymath}
   F(1, t) = \left\{
     \begin{array}{lr}
       G(1, 2t) & : t \in [0, \frac{1}{2}]\\
       H(1, 2t-1) & : t \in [\frac{1}{2}, 1]
     \end{array}
   \right.
   = \left\{
     \begin{array}{lr}
      f_0(1-2t) & : t \in [0, \frac{1}{2}]\\
       f_1(2t-1) & : t \in [\frac{1}{2}, 1]
     \end{array}
   \right. 
   =\overline{ f_0} \ast f_1.
\end{displaymath}
\end{document}