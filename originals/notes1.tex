\documentclass[12pt]{amsart}
\title{Topology Notes for 20 January 2010}
\author{Prepared by Arolyn Conwill}

\usepackage{amsmath,amsfonts,amssymb} 
\usepackage{geometry} 
\geometry{letterpaper,tmargin=1in,bmargin=1in,lmargin=1in,rmargin=1in}

\begin{document}

\begin{center}
\textbf{TOPOLOGY NOTES: 20 January 2010} \\
\textsc{Prepared by Arolyn Conwill}
\end{center}

\section{Introduction}
\subsection{Three Parts of the Course}
\begin{enumerate}
	\item Basic definitions and examples,
	\item Build new spaces from old spaces,
	\item Develop tools to distinguish spaces.
\end{enumerate}

\subsection{What is geometry?} Geometry is the study of rigid shapes that can be distinguished with measurements (length, angle, area, \ldots).

\subsection{What is topology?} Topology is the study of shapes which are equivalent via deformations. 

\subsubsection{Topology versus Geometry} Objects that have the same topology do not necessarily have the same geometry. For instance, a square and a triangle have different geometries but the same topology.
%FIGURE FIGURE FIGURE FIGURE FIGURE
\vspace{2cm}

\subsubsection{Motivation} Our goal is to understand the shape of our universe. Consider the following examples of two-dimensional universes: a plane, a sphere, a torus, and planes connected by tubes.
%FIGURE FIGURE FIGURE FIGURE FIGURE
\vspace{2cm}

\noindent These are topologically distinct universes. Intuitively, we can see that their ``holes'' distinguish them. Hence we seek a mathematical way to describe the holes; one familiar concept we will use is continuity, which is related to a lack of holes.

\section{Some Review from Analysis}

\subsection{Continuity} Let $f : \mathbb{R} \rightarrow \mathbb{R}$ and $a \in \mathbb{R}$. We say $f$ is continuous at $a$ if $\forall \epsilon > 0$ $\exists$ $\delta > 0$ such that if $|x-a|<\delta$, then $|f(x)-f(a)|<\epsilon$. 

\subsection{Metric Space} Let $M$ be a set and $d: M \times M \rightarrow \mathbb{R}$ be a function such that 
	\begin{enumerate}
		\item $d(a,b)=0$ iff $a=b$ (\textbf{nondegeneracy})
		\item $\forall a,b,c \in M$, $d(b,c) \leq d(a,b) + d(a,c)$ (\textbf{triangle inequality}).
	\end{enumerate}
Then we say $d$ is a \textbf{metric} or distance and $(M,d)$ denotes a \textbf{metric space}. 

\subsubsection{Compare this to the usual definition of a metric space} The above definition is equivalent to the usual definition of a metric space, but does not explicitly state the properties of positivity or symmetry:
	\begin{enumerate}
		\item $d(a,b) \geq 0$ $\forall a,b \in M$ (\textbf{positivity})
		\item $d(a,b) = d(b,a)$ $\forall a,b \in M$ (\textbf{symmetry}).
	\end{enumerate}

\subsubsection{The Usual Metric} The \textbf{usual metric} on $\mathbb{R}^{n}$ is
	\begin{displaymath}
		d((x_1,\ldots,x_n),(y_1,\ldots,y_n)) = \sqrt{\sum_{i=1}^n (x_i-y_i)^2}.
	\end{displaymath}

\subsubsection{The Discrete Metric} Let $M$ be any set. Then the \textbf{discrete metric} is defined as
	\begin{displaymath}
		  d(a,b) = \left\{
     						\begin{array}{lr}
       						1 & \text{if } a \neq b \\
       						0 & \text{if } a = b
     							\end{array}
   						 \right.
	\end{displaymath}
The discrete metric can be useful for testing conjectures as it does not rely on $\mathbb{R}^n$.

\subsubsection{The Comb Metric for $\mathbb{R}^2$} Let $X_0 = \{ 0 \} \times [0,1]$, $Y_0 = [0,1] \times \{ 0 \}$; and $\forall n \in \mathbb{N}$, let $X_n = \{ \frac{1}{n} \} \times [0,1]$. Let $M = (\cup_{n=0}^\infty X_n) \cup Y_0$. The distance is the distance measured along the comb in $\mathbb{R}^2$.  
%FIGURE FIGURE FIGURE FIGURE FIGURE
\vspace{5cm}

\noindent Using the comb metric,
\begin{itemize}
	\item Does the sequence $\{ (\frac{1}{n},0) \}$ converge? Yes, to the origin.
	\item Does the sequence $\{ (\frac{1}{n},a) \}$ converge when $a \in (0,1]$? No, since $d((\frac{1}{n},a),(0,a)) > 2a$ $\forall n$.
\end{itemize}

\subsubsection{A Non-Example of a Metric Space} Can you come up with a non-example of a metric space that satisfies the first property but not the second property? Consider the line of real numbers with $d(a,b) = a-b$. Clearly, $d(a,b)=0$ only when $a=b$, satisfying the first property (nondegeneracy). However, the second property (triangle inequality) is not satisfied: if $a=0$, $b=1$, and $c=-1$, then $2=d(b,c) \nleq d(a,b)+d(a,c)=-1+1=0$. Notice that this non-example also fails to satisfy the other two properties listed in the usual definition of a metric, positivity and symmetry.

\end{document}