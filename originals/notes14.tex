\documentclass{article}

\usepackage{amsthm}			%Add this one!!
\usepackage{amssymb}


\begin{document}
(continuation of the Long Example from previous class)

WTS: $\exists  U' \in F_{\pi \times i}$ such that $U' \notin F_{\frac{\mathbb{R}}{\sim} \times \mathbb{Q}}$

We want to construct $U$ as a union.

For all $n\in \mathbb{N}$ define $U_n$ to be the interior of the region of $\mathbb{R}\times\mathbb{Q}$ bounded by the vertical lines $n-\frac{1}{4}$ and $n+\frac{1}{4}$ and above and below by the lines through $(n, \frac{\sqrt{2}}{n})$ with slope $\pm 1$. Each one of these sets has a vertical line through the natural number.

So $\forall n \in \mathbb{N}, \{n\}\times \mathbb{Q} \subseteq U_n$. Observe that $\forall n\in \mathbb{N}, U_n \in F_{\mathbb{R}\times\mathbb{Q}}$ (it is an interior, so it is open).

Let $\bigcup_{n \in\mathbb{N}} U_n \in F_{\mathbb{R}\times\mathbb{Q}} \qquad$ (also open in $\mathbb{R}\times\mathbb{Q}$ because it is a union).

Let $U' = (\pi \times i)(U).$ This will glue together the strips along the vertical lines with $\mathbb{R}$ coordinates in $\mathbb{N}$. This can be pictured as a fan.

\smallskip

Claim: $U' \in F_{\pi \times i}$

{\it proof}: $(\pi \times i)^{-1}(U') = U$ and $U$ is a union of $U_n 's$

$\therefore$ by definition $U' \in F_{\pi \times i}$.

\smallskip

Claim: $U' \notin F_{\frac{\mathbb{R}}{\sim} \times \mathbb{Q}}$

{\it proof}: Let $([n],0) \in U'$. Suppose $\exists W \in F_{\frac{\mathbb{R}}{\sim}}$ and $V \in F_{\mathbb{Q}}$ such that $([n], 0)\in W \times V \subseteq U'$.

Then $\pi^{-1}(W) \in F_{\mathbb{R}}$ and of course $i^{-1}(V) = V$.

So $\pi^{-1}(W) \times V$ is open in $\mathbb{R}\times \mathbb{Q}$ such that $\pi^{-1}(W) \times V \subseteq (\pi \times i)^{-1}(U') = U$.

Also, $\pi^{-1}(W)$ is open in $\mathbb{R}$ and contains $\mathbb{N}$ and $V$ is open in $\mathbb{Q}$ and contains $\{0\}$.

So $\exists \delta >0$ such that $(-\delta, \delta) \cap \mathbb{Q} \subseteq V$. There exists an $n\in \mathbb{N}$ such that $\delta > \frac{\sqrt{2}}{n}$. $\pi^{-1}(W)$ is open in $\mathbb{R}$, so it's on the x-axis.

So $\exists \epsilon >0$ such that $(n-\epsilon, n+\epsilon)\subseteq \pi^{-1}(W)$.

So the interval $(n-\epsilon, n+\epsilon)\times (-\delta, \delta)\subseteq \pi^{-1}(W)\times V$. So $\epsilon < \frac{1}{4}$ from our previous definition of $U_n$.

Let $x = n + \frac{\epsilon}{2}$. Question: Where does $x = n + \frac{\epsilon}{2}$ meet the boundaries of $U_n$? It meets at $y = \frac{\sqrt{2}}{n} \pm \frac{\epsilon}{2}$.

$\exists y \in \mathbb{Q}$ such that $\frac{\sqrt{2}}{n} - \frac{\epsilon}{2} < y < \frac{\sqrt{2}}{n} + \frac{\epsilon}{2}$.

So $(x,y) \notin U_n$ and $\forall m\neq n, (x,y)\notin U_m$ since $(x,y) \in (n-\epsilon, n+\epsilon)\subseteq (n-\frac{1}{4}, n+\frac{1}{4}$.

But $(x,y) \in (n-\epsilon, n+\epsilon) \times (-\delta, \delta) \subseteq \pi^{-1}(W) \times V \subseteq U$ $\Rightarrow (x,y) \notin U$. Therefore we have a contradiction.

$\therefore U' \notin F_{\frac{\mathbb{R}}{\sim} \times \mathbb{Q}}$

So we have shown that $F_{\pi\times i} \neq F_{\frac{\mathbb{R}}{\sim} \times \mathbb{Q}}$. $\square$
















\end{document} 